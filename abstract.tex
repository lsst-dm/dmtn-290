
\begin{abstract}
At Rubin Observatory, we recently consolidated our high-frequency telemetry harness, which captures engineering facility data, with our metrics analysis system into a single unified service named Sasquatch. Within the LSST Science Pipelines, metrics are measured across various dimensions, such as instrument, detector, filter, exposure, dataset, and run, among others. With thousands of metrics measured during the 10-year survey, we will generate a high volume of time series data. Metrics and telemetry are both timestamped data, and to store and query this data efficiently we adopted InfluxDB, a specialized time series database. A crucial aspect of the architecture for capturing telemetry data is to ensure minimal data loss. This is achieved by deploying Kafka and InfluxDB on the Kubernetes platform. Sasquatch facilitates seamless real-time data access at the US Data Facility (USDF) through Kafka-based replication. Users access the data through Chronograf, an interface for time series visualization. They create alerts on the data using Kapacitor, and utilize the Rubin Science Platform's notebook environment for ad hoc analysis in Python. These tools have been extensively employed at Rubin during the ongoing System Integration Testing and Commissioning phase as  Rubin Observatory transitions into operations.
\end{abstract}

