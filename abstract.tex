
\begin{abstract}
At Vera C. Rubin Observatory, the need to manage metrics and telemetry data efficiently led to the creation of Sasquatch. Sasquatch consolidates our high-frequency telemetry harness, which captures the observatory engineering data, with the science performance metrics measured by the LSST Science Pipelines. Sasquatch utilizes InfluxDB, a time series database to efficiently store and query time-series data. For high availability and data durability we combine InfluxDB Enterprise with Apache Kafka deployed on the Kubernetes platform. The setup at the US Data Facility also enables real-time access of data mirrored from the Summit as well as historical data and provides tools like Chronograf for time series data visualization, Kapacitor for alert management, and the Rubin Science Platform’s notebook environment for data analysis using Python. Currently employed during our System Integration Testing and Commissioning phase, Sasquatch is proving essential as we transition into full operations, offering a scalable solution to our data management challenges.
\end{abstract}

